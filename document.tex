\documentclass[10pt,twocolumn]{article}

% use the oxycomps style file
\usepackage{oxycomps}

% usage: \fixme[comments describing issue]{text to be fixed}
% define \fixme as not doing anything special
\newcommand{\fixme}[2][]{#2}
% overwrite it so it shows up as red
\renewcommand{\fixme}[2][]{\textcolor{red}{#2}}
% overwrite it again so related text shows as footnotes
%\renewcommand{\fixme}[2][]{\textcolor{red}{#2\footnote{#1}}}

% read references.bib for the bibtex data
\bibliography{references}

% include metadata in the generated pdf file
\pdfinfo{
    /Title (Ethical Considerations of Computer Science Educational Resources)
    /Author (Julianne Yotov)
}

% set the title and author information
\title{Ethical Considerations of Computer Science Educational Resources}
\author{Julianne Yotov}
\affiliation{Occidental College}
\email{jyotov@oxy.edu}

\begin{document}

\maketitle

In today’s digital world, computer science education is becoming increasingly important, even for those that do not work in the technology industry. According to a report detailing the importance of computer science education in primary and secondary school around the world, “CS skills enable individuals to understand how technology works, and how best to harness its potential to improve lives. The goal of CS education is to develop computational thinking skills, which refer to the ‘thought processes involved in expressing solutions as computational steps or algorithms that can be carried out by a computer’ (K-12 Computer Science Framework Steering Committee, 2016)... Given the increasing integration of technology into many aspects of daily life in the 21st century, a functional knowledge of how computers work—beyond the simple use of applications—will help all students.”\cite{Brookings} Computer science, programming, and digital literacy classes are becoming increasingly commonplace in schools around the world, but they are still not nearly as universal as necessary. Only 53 percent of high schools in the United States offer computer science classes.\cite{Cal} Therefore, it is logical to deduce that a large number of students graduating high school are entering college or the workforce without any formal computer science education. My current comps project will ideally be one resource that people can turn to for support in introductory-level undergraduate computer science courses or as a way to learn and practice certain foundational aspects of coding. Users will be able to practice writing functions or methods and writing some test cases to test their code. However, it is a large responsibility to create such an educational resource, and there are many potential reasons why such a project should not be undertaken by one individual, but rather by a larger group of people with the time and resources to fully and effectively build such a resource. This paper explores the numerous ethical issues of creating this educational tool which raise the question of whether it is possible to complete this project while adequately addressing all of them.

Although my intention is that my project will be designed for beginners when it comes to coding and computer programming, I assumed a certain level of digital literacy when brainstorming this idea. However, this makes the resource inaccessible to a wide range of people. For instance, many high school graduates who did not take any computer science courses throughout their primary, secondary, and high school education may also not have had the necessary support systems to develop the digital literacy necessary to even begin learning how to code. Additionally, many people who completed their formal education before the existence of today’s technologies may not have the digital literacy necessary to undertake the process of learning about computer programming. According to a study on how older adults gain digital literacy via the use of tablet computers, “Even though there have been some recent gains in technology use, older adults are still the group the least likely to have crossed the digital divide… For example, in 2014, the overall Internet adoption rate in the United States was 87 [percent], while the adoption rate for people over 65 years old was only 57 [percent] (Pew Research Center, 2014).”\cite{Adults} These groups will face additional barriers to being able to make the most of an educational resource for coding compared to those that have benefited from a robust computer science and digital literacy education starting in primary school. Even if members of this latter group need support as they continue to build on their coding skills, they are more likely to know the existing resources available to them and how to overcome technology-related challenges. Additionally, they are more likely to have developed a certain amount of confidence when working with technology. Therefore, my project could very well be inaccessible to groups of people who perhaps need a free resource to learn foundational skills in computer programming, which is increasingly applicable to a large number of fields.

I would like to ensure that I provide a clear explanation of the purpose of my project and proper documentation for how to make use of the resource. This documentation can be modeled by the approach used for creating “datasheets for datasets.” The authors of this paper explain their idea: “In the electronics industry, every component, no matter how simple or complex, is accompanied with a datasheet describing its operating characteristics, test results, recommended usage, and other information. By analogy, we propose that every dataset be accompanied with a datasheet that documents its motivation, composition, collection process, recommended uses, and so on.”\cite{Datasheets} Even though I will not be working with a dataset, I can still write up documentation that includes my motivation behind the project, as well as the intended use and explanations of the different features. However, I will be writing this documentation independently, and that means that I likely need to rely extensively on user feedback to ensure that this documentation is clear and sufficient. This is in contrast to the large corporations that utilize datasheets, where numerous people, perhaps with different areas of expertise, are able to collaborate on the process of writing it. The paper goes on to provide guidelines on the different questions that a datasheet should cover, and it may simply not be possible for me to adequately address many of these questions in the timeframe of the semester. Additionally, I have biases that will impact how I approach the process of writing the documentation as well as its contents. While user feedback may help me gain a wider perspective on the thoughts of different individuals, it may be challenging to incorporate all of that feedback. The users themselves will also bring their own biases, and conflicting opinions will ultimately still result in me having to make the final decision about what is to be included in the documentation. Even when using the questions for the datasheets outlined in this paper, it may be challenging to decide which questions apply to my project, as well as which questions don’t directly apply but could be tweaked so that they do. The wording of almost all of the questions will have to be changed because I am not working with a dataset, and me independently rewriting the questions would once again result in the revised questions being written in a form that is most appealing to me and reflects my biases. Taking into account these challenges, it is unlikely that I will be able to write sufficient documentation for this educational resource. This will very likely have a negative impact on the experience of users, whether because the intended use and/or explanations of different aspects of the website are unclear or because the way I phrase my motivation behind the project seems unappealing to them.

Another important consideration is the effectiveness of virtual feedback on coding-based exercises. One of my major goals for my project is to effectively guide users through the process of writing the methods or functions, along with writing effective test cases, by providing extensive and effective feedback for errors. My experience with virtual tutoring in coding and other scientific disciplines has allowed me to provide solely verbal, individualized feedback to the students that I have worked with. A study focusing on the comparison of electronic and paper-based assignment submission and feedback found that “93 [percent] of students preferred having their feedback available online rather than printed and handed to them.”\cite{Feedback} This suggests that virtual feedback itself can be beneficial, but the quality of the feedback itself must be considered. When I worked with students one-on-one, I was able to provide individualized feedback while taking the student’s age and previous coding experience into consideration. This will not be possible when I do not know anything about the identity of the user, and therefore, the feedback will be generic and the same for all users. There are many questions to consider when thinking about the way in which feedback is presented. How can relatively generic feedback be used to help and encourage users of different backgrounds? How much should the feedback allow the user to figure out their error on their own? Additionally, something that encourages some individuals may discourage others, so it is difficult to ensure that everyone would receive the same benefit from receiving the feedback. Unfortunately, guidelines for providing effective virtual feedback differ based on the discipline, and it is challenging to determine which aspects are transferable to computer science. There seems to exist little literature discussing guidelines for effective feedback for coding assignments beyond autograders. One of the most significant challenges to creating this educational resource will be the process of creating the feedback that will be used to guide users through coding exercises. In the timeframe of a semester, it may simply be unrealistic to adequately consider all of the necessary questions that come with writing this feedback.

Educational resources require frequent maintenance to ensure that they continue to meet the needs of users. According to a paper on discussing an approach to website maintenance, “With the rapid evolutions of the Web applications, effective maintenance techniques to guarantee the correctness of the evolutions are highly demanded.”\cite{Maintenance} When creating the website, I will have to carefully consider how many exercises I want to include. However, I also want users to be able to utilize this resource for a certain period of time; if they are able to complete all of the exercises within a couple of days, it won’t have the lasting impact that I intend. This therefore likely means that in order for this to be an effective resource, it is important to continuously update it with new exercises that keep users engaged and ensure that they can make use of the resource over the course of an entire semester, or longer, depending on their pace. However, given I will work on my project over the span of a semester, it is unlikely that I will continue updating it with new iterations afterwards. This calls into question the effectiveness of my project, as my initial motivation for undertaking it was for it to be an effective educational resource for beginners in computer programming and coding. Therefore, it is likely that I will have to find some kind of balance between ensuring that my project can be completed within the required timeframe of a semester and including enough exercises to keep users engaged for a certain period of time. However, my lack of continuous maintenance likely means that this resource will not be as effective as existing resources that are moderated by large teams, such as Codecademy, LeetCode, GeekforGeeks, and W3Schools. My project will also include significantly less educational content outside of the coding exercises, so this calls into question what my project would accomplish that isn’t already covered by multiple other educational resources that are maintained frequently. In short, my project may best be left to a much larger team of individuals who have the time and resources to ensure that the website is actively maintained and continues to engage users. 


While computer science education very clearly requires technology, there is also the question of whether creating this educational resource is the most effective way (or even an effective) way to make it more accessible. As previously discussed, this resource may be more accessible to individuals who already have the resources, support, and continuous outside encouragement needed to study computer science, while remaining inaccessible to those who could most benefit from a free, self-paced, online educational resource. This could specifically include groups that are underrepresented in the technology industry. Without the proper support structures to encourage some students to study computer science, many may not even be aware of the extent to which online resources can effectively teach foundational computer programming concepts. According to a paper on the measures that can be taken to increase girls’ interest in “secondary computer science education”: “Despite interventions over the past decade, the gender gap in computing seems to be increasing and worldwide, almost everywhere fewer than one in five computer science graduates are female (Schlegel 2016). This urges fundamental changes in the way we approach early education of girls in computer science and STEM (Science, Technology, Engineering and Mathematics) (Gorbacheva et al. 2019), as the declining interest of girls in these disciplines seems to be preventable through tailoring education to girls’ specific needs to help them embrace computing (Accenture 2016).” The paper’s discussion section starts with, “Our aggregated umbrella review revealed some gaps in existing research on the topic (RQ4) [Are there gaps in the examined measures and strategies that would deserve better research coverage?], which are connected to the segregation of causes and solutions in existing studies, as well as missing easy-to-use tools and guidelines on gender-sensitive CS education, i.e., consciously working with the awareness that each education style might have different effectiveness for each gender (Diller 2018).”\cite{GirlsinTech} A motivation behind my project is that it would be accessible to a wide range of people, and especially those who continue to be underrepresented in higher computer science education. However, this may very well not be the case. I will likely not be able to do enough research on how to build my website in a way that provides encouragement to students who unfortunately do not receive it from their schools and teachers. Therefore, my website might end up working to benefit those who already have the resources to pursue or continue pursuing computer science education, not serving those who may be most in need of an accessible resource.

Creating an educational resource is a large responsibility, and it is not feasible for me to adequately address all of the ethical considerations presented in this paper. This then leads to the conclusion that the important task of continuing to build and maintain an educational resource for computer science is best left up to a team of people with the expertise, time, and resources to ensure that they are doing everything possible to make this resource as accessible and user-friendly as possible. A large portion of the semester will be devoted to me actually building the website and conducting user research, which unfortunately will leave little to no time for creating numerous iterations of the documentation and features of the website. Even through extensive user research, the people that I interview will not necessarily have the same thoughts and experiences as students outside of Occidental. This group will probably consist of individuals who all have access to technology, have a certain level of digital literacy, all fall within a narrow age bracket, and are all college students. Therefore, while their feedback and personal experiences with computer science education will be unique, individuals who do not have the same access to technology, are significantly younger or older, and are not yet college students or haven’t had the same educational background may have vastly different thoughts. Even when incorporating the feedback that I do receive, I must not view this as effectively having taken all possible ethical considerations into account. Even if this project can be done ethically by a large enough group of people, it is not possible for me to effectively address all of the ethical concerns within the timeframe of a semester to a sufficient extent.  

\printbibliography

\end{document}
